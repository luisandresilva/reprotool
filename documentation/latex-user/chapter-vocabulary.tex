\section{Modifying the annotation vocabulary and semantics}

Right-click the project file for which you want to modify the annotation semantics and open it with the \emph{Swproj Model Editor}.
There you can view or edit the EMF model objects that build up a \emph{Reprotool} project. Under the \emph{Software Project} tree node,
you can see the default \emph{Annotation Groups}, what \emph{annotations} and \emph{CTL formulas} they contain. By modifying these
formulas, you can modify the annotation semantics. You can also add new \emph{AnnotationGroups} to add custom annotations to the
\emph{Reprotool} project and to specify their semantics using the NuSMV \emph{CTL} or \emph{LTL} formulas.

\begin{figure}[ht]
  \centering
  \includegraphics[height=280pt]{images/reprotoolCTLFormulas}
  \caption{The \emph{open-close} annotation group with annotations and default \emph{CTL} formulas}
  \label{fig:reprotoolCTLFormulas}
\end{figure}

\newpage

\subsection{Adding \emph{lock / unlock} annotations to a \emph{Reprotool} project}
Now we are going to show you how you can add new annotations to a \emph{Reprotool} project. We will add two annotations to our project.
These two annotations are \emph{lock} and \emph{unlock}. These two annotations will use the semantics of a recursive lock. More
precisly, the model checking in \emph{Reprotool} will check if there is an \emph{unlock} annotation present in every possible
execution path after the \emph{lock} annotation appears.

\subsubsection{Adding a new \emph{temporal annotation group} to the project}
First we need to add a new \emph{temporal annotation group} to our project. Open the project using \emph{Swproj Model Editor}.
Right-click on the \emph{Software project} tree node and select the \texttt{New Child / Temporal Annotation Group} command.
This will add a new \emph{temporal annotation group} to our project.

\subsubsection{Adding annotations to the \emph{temporal annotation group}.}
Now we are going to add the \emph{lock} and \emph{unlock} annotations to the newly-created annotation group. Right-click this annotation
group and select the command \texttt{New Child / Temporal Annotation}. Fill the annotation name as \emph{lock} in the properties
view. Use the same command to add the \emph{unlock} annotation to this annotation group.

\subsubsection{Adding semantics to the \emph{lock / unlock} annotations.}
Right-click the new \emph{Annotation group} in the \emph{Swproj Model Editor} and select the command \texttt{New Child / CTL Formula}
from the context menu. Now type \emph{After 'lock' there always needs to be 'unlock'} as the formula \emph{Description} in the
\emph{Properties view}. Type \texttt{AG(lock -> AF(unlock))} as \emph{Formula} in the \emph{Properties view}. You will see in the
next subsection how to build your own NuSMV \emph{CTL} formulas.
You should now be in a situation shown in the following figure:

\newpage

\begin{figure}[ht]
  \centering
  \includegraphics[height=280pt]{images/reprotoolLockUnlock}
  \caption{The new \emph{lock-unlock} annotation group}
  \label{fig:reprotoolLockUnlock}
\end{figure}

In the \emph{Reprotool} example projects, there is a project named \emph{Lock example}. We have added the \emph{lock / unlock}
annotations to this project. In the project, you can find a simple use-case that has a single execution path where the resources
are not properly unlocked. Running the model-cheking process on this example yields the expected result:

\begin{figure}[ht]
  \centering
  \includegraphics[height=280pt]{images/reprotoolLockUnlock2}
  \caption{The \emph{unlock} annotation is missing in this execution path}
  \label{fig:reprotoolLockUnlock2}
\end{figure}

\newpage

\subsubsection{CTL formulae operators}
We use in \emph{Reprotool} the following operators to build up our CTL formulas. (The descriptions of operators are taken from
the {nusmv-manual} that you can find in the \emph{Reprotool} svn documentation directory.) 

\begin{itemize}
 \item \textbf{AG $p$} is \emph{true} in a state $s_{0}$ if for \emph{all} infinite series of transitions $s_{0} \rightarrow s_{1} , s_{1} \rightarrow 
  s_{2} , \ldots$ \emph{p} is \emph{true} in \emph{every} $s_{i}$. (\emph{forall globally})
 
 \item \textbf{AF $p$} is true in a state $s_{0}$ if for all series of transitions $s_{0} \rightarrow s_{1}, s_{1} \rightarrow
  s_{2},\ldots, s_{n-1} \rightarrow s_{n}$ \emph{p} is \emph{true} in $s_{n}$. (\emph{forall finally})

 \item \textbf{EF $p$} is \emph{true} in a state $s_{0}$ if there \emph{exists} a series of transitions $s_{0} \rightarrow s_{1}, 
 s_{1} \rightarrow s_{2},~\ldots, s_{n-1} \rightarrow s_{n}$ such that \emph{p} is \emph{true} in $s_{n}$
  
\item \textbf{AX $p$} is true in a state \emph{s} if for all states $s'$ where there is a transition from \emph{s} to
  $s'$ , \emph{p} is \emph{true} in $s'$. (\emph{forall next state})
  \item \textbf{A[$p \cup q$]} is \emph{true} in a state $s_{0}$ if for \emph{all} series of transitions $s_{0} \rightarrow s_{1},
  s_{1} \rightarrow s_{2},~\ldots, s_{n-1} \rightarrow s_{n}$ \emph{p} is \emph{true} in \emph{every} state from $s_{0}$ to
  $s_{n-1}$ and \emph{q} is \emph{true} in state $s_{n}$. (\emph{forall until})
\end{itemize}

\subsubsection{CTL formulas examples}
\begin{description}
 \item[$AG(open \rightarrow AF(close))$] After 'open' there should always be 'close'
 \item[$AG(open \rightarrow AX(A\lbrack!open \cup close\rbrack))$] No multi-open without close
 \item[$AG(close \rightarrow AX(A\lbrack!close \cup open \mid !AF(close) \rbrack))$] No multi-close without open
 \item[$A\lbrack !close \cup open \mid !AF(close)\rbrack$] First 'open' then 'close'
 \item[$AG( create \rightarrow EF(use) )$] After 'create' there must be some branch containing 'use'
 \item[$AG( create \rightarrow AX(AG(!create)) )$] Only one 'create'
 \item[$A\lbrack !use \cup create \mid !AF(use)\rbrack$] No 'use' before 'create'
\end{description}